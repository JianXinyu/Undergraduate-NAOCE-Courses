\section{波动现象}
	\subsection{Summary}
	 是海水运动的主要形式之一,具有巨大的能量,一种复杂的自然现象,由多种自然因素引起.
	\begin{enumerate}
		
		\item
		海浪=风浪+涌浪 
		
		风浪:风直接作用下产生的水面波动; 
		
		涌浪:风浪离开作业海区后或风速急剧下降后由重力惯性继续作用而成.
		
%		\item g海浪的运动及形态因所处水深不同而不同,深水区内传播的波浪形态基本不变,浅水区内因受海底摩擦影响,水分子作椭圆运动,产生波浪折射、绕射、反射或破碎等现象。
		
		\item 海浪的研究手段:理论方法,实验模拟,现场观测.
		
		\item 海浪灾害. 海浪要素是海洋工程结构物在设计施工中必须考虑的 \textbf{环境载荷条件} 之一.海浪要素:波长周期波速波高振幅,波陡,波峰线,波向线-与波峰线相垂直的线,表示波动传播方向.
		
		 不规则波的海浪要素,采用上跨零线相交法定义.周期,平均周期.波高(平均波高,有义波高,十分之一大波波高,累计率波高,最大波高).波长.
		 
		\item 海浪要素分布:1)深水波高服从瑞利分布.2)
	\end{enumerate}
		
	\subsection{波浪理论}
		运动方程+ 连续方程+ 边界条件;
		假定:流体无粘,运动无旋,波面压力为常数;
		振幅波长比 $a/\lambda$ 大小作不同处理.
		\begin{enumerate}
			\item 小振幅重力波:波动振幅相对波长为无限小,$a/\lambda<<1$.重力是其唯一外力,\textbf{线性}理论结果.
				\begin{enumerate}
					\item 前进波:\textbf{波高,波数,波速,圆频率,色散关系}:水深一定,不同波长的水波以不同速度传播而导致波的分散现象.$\omega^2 = kgtanh(kh)$. \textbf{波动总能量,分类,椭圆/圆}
					\item 驻波:波型不向外传播
					\item 波群:许多频率不同的波叠加在一起而在波面上形成波群,以群速传播
									
				\end{enumerate}
%			\item 有限振幅波动理论
%				\begin{enumerate}
%					\item Stokes波:振幅波比不是小量,意义
%					\item 椭圆余弦波:浅水区域计入水深影响
%					\item 孤立波
%					\item 深水->线性波和Stokes高阶波,浅水->椭圆余弦波和孤立波
%				\end{enumerate}
		\end{enumerate}
		
	\subsection{随机过程的海浪}
		\begin{enumerate}
			\item 海浪谱:只用于是说明海浪能量相对于频率分布的谱
		\end{enumerate}
		
	\subsection{浅海近岸的海浪特性}
		\begin{enumerate}
%			\item 波长波速的变化:$c/c_0=\lambda/\lambda_0=tanh(2\pi h\lambda)$
%			\item 波向的折射
%			\item 波高的变化
 			\item 海浪的破碎:崩破波,卷破波,激破波
%			\item 离岸流与沿岸流
%			\item 反射
%			\item 绕射
%			\item 浅水波长,波速,波高计算
		\end{enumerate}
	
	\subsection{海浪观测及中国近海波浪}
		海浪观测的\textbf{主要项目}:浪高,波长,周期,波速,波型,波向,海况.观测方式\textbf{目测,仪测},观测海浪时,应同时观测\textbf{风速,风向,水深}
		\begin{enumerate}
			\item 海浪玫瑰图
			\item 中国近海波浪特性:1)季节性强:冬季盛行偏北风->偏北浪,夏季盛行偏南风->偏南浪,春秋为其过渡季节;冬春季多寒潮和冷空气,夏秋季多台风和热带气旋;造成沿海波况以寒潮浪及\textbf{台风浪}为主.2)浪高分布特性:冬季平均浪高最大,夏季低;冬季风浪周期最大,夏季小;北部小,南部大.
			
		\end{enumerate}
	
	\subsection{波浪作用力}
		
	\subsection{海啸,风暴潮,内波}
		\begin{enumerate}
			\item 海啸由海底地震,火山爆发,塌陷和滑坡引起.特点:突发难预测,作用距离远,危害大.
			\item 风暴潮是我国海洋灾害之首,分为台风风暴潮和温带风暴潮.阶段:先行涌浪->风暴潮->余震阶段. 移动至近岸浅水区水位急剧增高.\textbf{风暴潮+天文大潮+降水}->海面水位暴涨,侵入内陆造成危害
			\item 内波是发生在海水密度层结稳定的海洋中的一种波动,最大振幅出现在海面以下.产生条件:存在稳定密度分层,有扰动源.
		\end{enumerate}
		