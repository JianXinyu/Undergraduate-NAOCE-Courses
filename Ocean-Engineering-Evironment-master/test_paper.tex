\section{testpaper}
名词解释3*3:1.跃层:水温要素在铅直方向上出现跃变的水层,分为大洋主温跃层,季节性温跃层  2.波 (波向线,波群) 3.涨 涨潮:一个潮汐周期内潮位上升的过程

填空14:1.中国近海 
		\begin{enumerate}
			\item 海洋综合调查与评价项目. 
			\item 温,盐,密度的特征. 
			\item 波浪.
			\item 潮汐与潮流. 
			\item 环流. 
			\item 水色和透明度的分布变化
		\end{enumerate}

2.我国将海域冰 
	 1.渤黄东南
	 2.根据我国海区的结冰特点,我国依据冰区范围和海冰厚
	 度将海域冰情划分为五个等级1. 轻冰年
	 2. 偏轻年
	 3. 常年
	 4. 偏重年
	 5. 重冰年

3.海洋腐蚀 环境及区域划分

选择(单/双)4*2: 
1. a.沿逆时针 b.沿顺时针 c.都可  
梯度风,对应气旋效应显著的风场.背风而立,高压在右,低压在左.绕高压中心作顺时针方向运动,绕低压中心作逆时针方向运动。

2.a.增大 b.缩小 c.增强 d.减弱  

抗压强度随加载率先增大再减小
抗弯刚度为随温度降低而增强,随\textbf{盐水体积}的增大而减弱
弹性模量随温度降低而增大,随盐水体积增大而减小

3.a.日最大风速 b.月最大风速 c.年最大风速
最大风速样本的取法影响着平均风速的数值,宜取 年最大风速 作为样本进行长期预测。

判断,给理由,3*3: 1.低.对吗?why 2.  3.风.对吗.why

简答3*5: 1.某处有何特点 2.波?简述 (波向线,波群,波浪破碎类型)3.对于?why

计算2*15: 1.某平台  2.计算水深

论述15:你认为.有何建议


2018.1.10 部分考题

\begin{enumerate}
\item 填空:
\begin{enumerate}
	\item 全球平均盐度
	\item 南半球潮流对海岸的侵蚀
	\item 根据我国海区的结冰特点,我国依据冰区范围和海冰厚
	度将海域冰情划分为五个等级1. 轻冰年
	2. 偏轻年
	3. 常年
	4. 偏重年
	5. 重冰年
\end{enumerate}

%\item 选择
%\begin{enumerate}
%	\item 
%\end{enumerate}

\item 判断
	\begin{enumerate}

		\item 黑潮影响我国海域.
	\end{enumerate}

\item 简答
	\begin{enumerate}
 
		\item 秘鲁上升流成因及影响
	\end{enumerate}

\item 计算
	\begin{enumerate}
		\item 深水波计算
		\item 风速换算及阻力计算
	\end{enumerate}

\item 论述:对船舶污染的认识和建议
\end{enumerate}