\section{海冰}
\subsection{海冰概述}
\subsubsection{概述}
\textbf{海冰是由海冰冻结而成的咸水冰,包括流入海洋的河冰和冰山等。}
占大洋面积的3-4\%
海冰和冰山是\textbf{高纬度}地区特有的海洋水文现象
\subsubsection{海冰灾害}
对交通运输生产作业海上设施和海岸工程等造成严重的灾害
封锁港口和航导,破坏运输船舶和海洋工程设施,影响海上运输安全和工业、渔业生产。
注意寒冷海域需要考虑\textbf{海冰载荷}
\subsubsection{对我国的影响}
发生在渤海和黄海北部
11底由北向南结冰,2月至3月由南向北消失
辽东湾最严重
\subsubsection{冰情划分}
用于比较和反映海上结冰的轻重及影响程度
指结冰范围、海冰厚度、海冰类型及其分布
\begin{enumerate}[1)]
	\item 轻冰年
	\item 偏轻年
	\item 常年
	\item 偏重年
	\item 重冰年
\end{enumerate}
\subsubsection{我国特点}
一年冰
有初冰期、盛冰期、终冰期
\subsubsection{重要性}
设计结构考虑海冰作用力

海冰作用力大小与海冰的类型、几何尺度、冰速冰向、冰的物理作用力等性质密切相关
\subsection{研究手段}
现场观测
试验研究
理论研究
\subsection{形成过程}
形成条件: 海水温度降至冰点并继续失热,相对冰点有过冷却现象并有凝结核存在
\subsubsection{盐度与密度}
结冰过程:纯水结成冰晶,将所含盐分排析出。未来得及排出部分则形成盐水泡,还含有气体
海冰的盐度随冰厚增加而减小,冰龄越长,冻结得越厚则冰层中的平均盐度越低,密度越小
\subsubsection{海冰类型}
按结冰过程的\textbf{发展阶段}:
初生冰:最初形成的海冰,针状或薄片状的细小冰晶;大量冰晶凝结,聚集成黏糊状或海绵状病。
尼罗冰:初生冰继续增长,冻结成厚度10cm左右有弹性的薄冰层,易弯曲,易被折碎成长方形冰块
饼状冰:破碎的薄冰片,外力作用下互相碰撞、挤压、边缘上升。
初期冰:有尼罗冰或冰饼直接冻结形成。
一年冰:由初期冰发展而成的厚冰,时间不超过一个冬季
老年冰:至少经过一个夏季而未融化的冰
按运动形态划分:
固定冰,
流浮冰,
按外形划分,
平整冰,
重叠冰,
堆积冰,
冰丘,
冰山
\subsection{物理力学特性}
物理力学力学特性包括冰的晶体结构、厚度、含盐度、抗压强度和抗弯强度
有明显的区域性。

作用在结构物上的冰压力主要取决于海冰自身的强度极限。
海冰的破坏强度等于海冰作用在结构的最大静冰力。
\subsubsection{抗压强度}
影响因素:海冰类型、温度、盐度、密度、加载速率、加载方向等

\textbf{加载率}影响很大,抗压强度随加载率先增大再减小
\subsubsection{抗弯强度}
缺乏完整的试验资料

主要因素表现为随温度降低而增强,随\textbf{盐水体积}的增大而减弱
\subsubsection{剪切强度}
一般,\textbf{卤水体积}增大,剪切强度减小。
\subsubsection{弹性模量}
是温度、盐度的函数
随\textbf{温度}降低而增大,随\textbf{盐水体积}增大而减小。

\subsection{海冰作用力}

\subsubsection{海冰的破坏形式}
\begin{enumerate}
	\item \textbf{挤压破坏}:海冰\textbf{垂直}于桩柱时,因海冰的抗压强度小于结构物的应力而被破坏。
	\item \textbf{弯曲破坏}:海冰沿结构物的\textbf{倾斜面}作用时,海冰抗弯强度小而受弯破坏
	\item \textbf{屈曲破坏}: 当厚度较小的\textbf{薄}冰与结构物发生挤压时,海冰\textbf{丧失稳定}性而被破坏
	\item 其他破坏 剪切破坏、剥落破坏、蠕变破坏
\end{enumerate}
\subsubsection{海冰作用力}
\begin{enumerate}
	\item 海冰的\textbf{水平挤压力}:风或海流等作用下的危险受力,对平台直接作用\textbf{最大}。
	\item 海冰的撞击力
	\item 海冰的垂向附着力
	\item 海冰的重力
	\item 冰层膨胀挤压的膨胀力与结构物之间的摩擦力
\end{enumerate}
\subsubsection{海冰对\underline{锥形结构物}的作用力}
海冰作用于倾斜结构时,冰排将沿斜面上移而弯曲,海冰将以受弯形式先发生破坏。
在海冰作用区采用锥形形式有利于减小海冰作用力。
\subsubsection{冰振现象}
冰振普遍存在,引起\textit{强度和疲劳}问题
对冰速敏感
与结构变性密切相关,\underline{柔性}结构更易发生自激振动
桩腿直径越小越易发生自激振动,冰越厚也越易发生。